\section{Conclusioni e Sviluppi Futuri}

Questo lavoro ha presentato la progettazione e l'implementazione di \textbf{AgroManager}, un sistema software per la gestione delle attività agricole. È stata seguita una metodologia strutturata, partendo dall'analisi dei requisiti (\hyperref[sec:analisi_requisiti]{Capitolo 2}) fino alla progettazione dettagliata (\hyperref[sec:progettazione]{Capitolo 3}) e all'implementazione dei componenti chiave (\hyperref[sec:dettagli_implementativi]{Capitolo 4}).

L'architettura a layer, unita all'adozione rigorosa di design pattern come il \texttt{Template Method} nel DAO layer e lo \texttt{Strategy Pattern} nel service layer, ha permesso di costruire un'applicazione robusta, manutenibile e disaccoppiata. La gestione centralizzata degli errori tramite una gerarchia di eccezioni custom ha ulteriormente contribuito alla solidità del sistema.

\subsection{Sviluppi Futuri}
L'architettura attuale, basata su principi di modularità e separazione delle responsabilità, pone solide fondamenta per numerose estensioni future. Alcuni dei possibili sviluppi includono:

\begin{itemize}
	\item \textbf{Autenticazione Utenti e Gestione Ruoli:} Introduzione di un sistema di login per differenziare gli utenti (es. Amministratore, Operatore). Questo permetterebbe di implementare controlli di accesso granulari, limitando l'accesso a determinate funzionalità (es. solo l'amministratore può gestire le zone, mentre l'operatore può solo registrare i raccolti).

	\item \textbf{Estensione del Modulo di Analisi:} Grazie all'uso dello \textbf{Strategy Pattern} (\hyperref[sec:strategy_pattern]{descritto nella Sezione 4.2.4}), è possibile aggiungere facilmente nuove strategie di calcolo e report. Si potrebbero implementare analisi più complesse, come la previsione della data di raccolta basata su dati storici, o un'analisi dei costi per zona agricola.

	\item \textbf{Integrazione IoT e Monitoraggio Live:} L'architettura \texttt{Service}-based è predisposta per ricevere dati da fonti esterne. Un'evoluzione significativa sarebbe l'integrazione con sensori IoT (umidità, temperatura, pH del terreno) che, tramite un servizio di background, aggiornano lo stato delle piantagioni in tempo reale.

	\item \textbf{Applicazione Mobile per Operatori:} Sviluppo di un'interfaccia mobile (es. Android o PWA) che consuma gli stessi \texttt{Service} già sviluppati. Questo permetterebbe agli operatori sul campo di registrare i raccolti o segnalare problemi direttamente da smartphone, migliorando notevolmente l'efficienza operativa.
\end{itemize}

In conclusione, il sistema AgroManager, pur essendo un prototipo funzionale, è stato progettato con un'attenzione particolare all'estensibilità, dimostrando come il sistema possa supportare il ciclo di vita e l'evoluzione di un prodotto complesso.